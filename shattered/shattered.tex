\documentclass{beamer}
\usetheme{Rochester}
\mode<presentation>

\title{SHAttered}

\date{March 2017}

\begin{document}

\begin{frame}
 \frametitle{First known SHA-1 collision}
 \begin{itemize}
  \item On 23 Feb 2017, Google announced the first SHA-1 collision
        attack (\url{https://shattered.io/}) \pause
  \item Two different PDFs, same SHA-1 hash \pause
  \item Effort: 6500 CPU-years \emph{and} 110 GPU-years \pause
  \item Countermeasures: counter-cryptanalysis \pause
  \item Online tool to generate colliding PDFs \pause
  \begin{itemize}
   \item \url{https://github.com/nneonneo/sha1collider/}
  \end{itemize}
 \end{itemize}
\end{frame}

\begin{frame}
 \frametitle{Demonstration}
 \begin{itemize}
  \item Original PDFs
  \item Generated PDFs
 \end{itemize}
\end{frame}

\begin{frame}
 \frametitle{Some theory}
 \begin{block}{Pre-image resistance}
  Given $y$, it is infeasible to compute $x$ such that $h(x) = y$
 \end{block} \pause
 \begin{block}{Second pre-image resistance}
  Given $x$, it is infeasible to compute $x'$ such that $h(x) = h(x')$
 \end{block} \pause
 \begin{block}{Collision resistance}
  It is infeasible to compute $x$ and $x'$ such that $h(x) = h(x')$
 \end{block} \pause
 \begin{itemize}
  \item This is an instance of a \emph{collision attack}: find two
        different files that produce the same hash
 \end{itemize}
\end{frame}

\begin{frame}
 \frametitle{Consequences}
 \begin{itemize}
  \item Crash SVN \pause
  \item Git? \pause
  \item Don't use SHA-1 anymore
 \end{itemize}
\end{frame}

\end{document}
