\documentclass{beamer}
\usetheme{Rochester}
\mode<presentation>

\usepackage{tikz-dependency}
\usepackage{tikz-qtree}
\usepackage{pgfplots}
\usepackage{verbatim}

% Vectors
\newcommand{\vect}[1]{\underline{#1}}

% Matrices
\newcommand{\mat}[1]{\mathbf{#1}}

% Differentials
\newcommand{\ud}{\, \mathrm{d}}
\newcommand{\udf}{\mathrm{d}}

% Sets
\newcommand{\sbar}{\, | \,}
\newcommand{\st}{\, . \,}

% Inner product
\newcommand{\innerprod}[2]{\langle #1,#2 \rangle}

% Norm
\newcommand{\norm}[1]{||#1||}

% Numerical coding of pairs
\newcommand{\fab}[2]{\langle \langle #1,#2 \rangle \rangle}
\newcommand{\ab}[2]{\langle #1,#2 \rangle}
\newcommand{\enc}[1]{\lceil #1 \rceil}

% Configuration
\newcommand{\conf}[2]{\langle #1,#2 \rangle}

% Hoare triple
\newcommand{\pcs}[3]{\{#1\} \ #2 \ \{#3\}}
\newcommand{\tcs}[3]{[#1] \ #2 \ [#3]}

% Discrete sequence
\newcommand{\dsq}[1]{\{#1\}}

% Braket
\newcommand{\bra}[1]{\langle#1|}
\newcommand{\ket}[1]{|#1\rangle}
\newcommand{\bk}[2]{\langle#1|#2\rangle}

% Argmin/max
\DeclareMathOperator*{\argmax}{arg\!\max}
\DeclareMathOperator*{\argmin}{arg\!\min}


\title{Algorithms and Data Structures}

\author{Darren Foong}

\date{January 2017}

\begin{document}

\begin{frame}
 \titlepage
\end{frame}

\begin{frame}
 \frametitle{Overview}
 \tableofcontents
\end{frame}

\section{Searching}

\begin{frame}
 \frametitle{Git}
  \visible<2->{
   \begin{block}{Problem}
    Find $n$ such that commit $n$ is ``good" and commit $n+1$
    is ``bad", i.e.\ commit $n+1$ introduced the error.
   \end{block}
  }

  \vspace{1cm}

  \visible<1->{
   \begin{center}
   \scalebox{1.2} {
    \begin{tikzpicture}
     \draw (0,0) -- (8,0);

     \foreach \i [evaluate=\i as \x using int(\i+50)]  in {0,...,8} {
      \fill[blue] (\i cm, 0) circle (4pt);
      \node at (\i cm, -12pt) {$\x$};
     }

     \fill[green] (0 cm, 0) circle (4pt);
     \fill[red] (8 cm, 0) circle (4pt);

     \pause
     \pause
     \fill[red](4 cm, 0) circle (4pt);

     \pause
     \draw[decoration={brace,raise=5pt},decorate] (1 cm, 0) -- (3 cm, 0) node[midway,yshift=0.5cm] {possibly bad};
     \draw[decoration={brace,raise=5pt},decorate] (5 cm, 0) -- (7 cm, 0) node[midway,yshift=0.5cm] {definitely bad};

    \end{tikzpicture}
   }
   \end{center}
  }
\end{frame}

\begin{frame}[fragile]
 \frametitle{\texttt{git bisect}}
 \begin{itemize}
  \item Linear search takes $7$ steps in the worst-case $\Rightarrow O(n)$
  \item Binary search takes $\approx \lg_2 7 \approx 3$ steps in the worst-case $\Rightarrow O(\lg n)$
  \item \texttt{git} has in-built binary search:
  \begin{verbatim}
$ git bisect start
$ git bisect good
$ git bisect bad
...
$ b41e... is the first bad commit
  \end{verbatim}
 \end{itemize}
\end{frame}

\begin{frame}
 \frametitle{Binary search trees}
 \begin{itemize}
  \item A binary tree with \emph{ordered keys}
   \begin{itemize}
    \item left child is less than parent
    \item right child is greater than parent
   \end{itemize}
  \item \emph{In-order traversal} returns keys in sorted order
  \item Time complexity for searching, insertion, and deletion is
        proportional to \emph{height} of the tree; $O(\lg n)$ for a
        balanced tree
 \end{itemize}
 \begin{center}
  \scalebox{1.0} {
   \begin{tikzpicture}[every tree node/.style={minimum width=0.5cm,
                                               draw,
                                               circle},
                       edge from parent/.style={draw,
                                                edge from parent path={
                                                 (\tikzparentnode) --
                                                 (\tikzchildnode)}},
                       sibling distance=1.0cm,
                       level distance=1.5cm]
    \Tree [.4 [.2 [.1 ] [.3 ] ] [.6 [.5 ] [.7 ] ] ]
   \end{tikzpicture}
  }
 \end{center}
\end{frame}

\begin{frame}
 \frametitle{Pathological cases}
 \begin{itemize}
  \item A binary search tree can end up unbalanced depending on the
        order of operations
  \item Insert na\"ively $1,2,\ldots,7$ in order:
 \end{itemize}
 \begin{center}
  \scalebox{0.5} {
   \begin{tikzpicture}[every tree node/.style={minimum width=0.5cm,
                                               draw,
                                               circle},
                       edge from parent/.style={draw,
                                                edge from parent path={
                                                 (\tikzparentnode) --
                                                 (\tikzchildnode)}},
                       sibling distance=1.0cm,
                       level distance=1.5cm,
                       blank/.style={color=gray,dashed}]
    \Tree [.1 \edge[blank]{}; \node[blank]{};
              [.2 \edge[blank]{}; \node[blank]{};
                  [.3 \edge[blank]{}; \node[blank]{};
                      [.4 \edge[blank]{}; \node[blank]{};
                          [.5 \edge[blank]{}; \node[blank]{};
                              [.6 \edge[blank]{}; \node[blank]{};
                                  [.7 ] ] ] ] ] ] ]
   \end{tikzpicture}
  }
 \end{center}
\end{frame}

\begin{frame}
 \frametitle{Self-balancing binary search trees}
 \begin{itemize}
  \item Performs rebalancing (``housekeeping") when carrying out
        operations to ensure height of tree is $O(\lg n)$
  \item Examples:
  \begin{itemize}
   \item Red-black trees (Java \texttt{TreeMap}, \texttt{TreeSet},
         C++ \texttt{std::map}, \texttt{std::set})
   \item AVL trees
  \end{itemize}
 \end{itemize}
\end{frame}

\section{Big-O notation}

\begin{frame}
 \frametitle{Big-O notation}
\end{frame}

\section{Sorting}

\begin{frame}
 \frametitle{Insertion sort}
\end{frame}

\begin{frame}
 \frametitle{Merge sort}
\end{frame}

\begin{frame}
 \frametitle{Quick sort}
\end{frame}

\begin{frame}
 \frametitle{Java implementation}
\end{frame}

\section{Maps}

\begin{frame}
 \frametitle{Search trees}
\end{frame}

\begin{frame}
 \frametitle{Hash tables}
\end{frame}

\begin{frame}
 \frametitle{Database indexing}
\end{frame}

\section{Priority queues}

\begin{frame}
 \frametitle{Heap property}
\end{frame}

\begin{frame}
 \frametitle{Other implementations}
\end{frame}

\section{Caveats}

\begin{frame}
 \frametitle{Hidden constant}
\end{frame}

\begin{frame}
 \frametitle{Worst-case analysis}
\end{frame}

\section{How to tackle a problem}

\begin{frame}
 \frametitle{Black box abstraction}
\end{frame}

\section{Interesting algorithms}

\begin{frame}
 \frametitle{Bloom filter}
\end{frame}

\begin{frame}
 \frametitle{Knapsack problem}
\end{frame}

\begin{frame}
 \frametitle{Point in polygon}
\end{frame}

\begin{frame}
 \frametitle{Why is grep so fast?}
\end{frame}

\section{Conclusions}

\begin{frame}
 \frametitle{What I did not cover}
 \begin{itemize}
  \item Lists, stacks, queues, sets
  \item Amortised analysis
  \item Graph algorithms
  \item Graphics algorithms
  \item and a lot more
 \end{itemize}
\end{frame}

\begin{frame}
 \frametitle{References}
 \begin{itemize}
  \item \emph{Introduction to Algorithms} by Cormen, Leiserson, Rivest, and Stein (CLRS)
  \item Cambridge Computer Laboratory courses: \url{https://www.cl.cam.ac.uk/teaching/current/}
  \item \url{https://github.com/darrenfoong/sqc-talks/}
 \end{itemize}
\end{frame}

\end{document}
